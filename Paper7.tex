\documentclass[12pt,a4]{article}
\usepackage{geometry}
\usepackage{graphicx}

\geometry{margin=1in}

\title{Influence of Multi-Axial Stress States on the Flexural Behaviour of Circular Concrete-Filled Steel Tubes}
\author{Yadong Jiang \and António Silva \and Bledar Kalemia \and José Miguel Castro \and Ricardo Monteiro}


\date{}

\begin{document}
	\maketitle
	
	\begin{abstract}
		The paper proposes the methodologies and the equations to consider the influence of multi-axial stress states on the material strength of circular Concrete-filled Steel Tubular (CFST) columns. From the previous experimental and numerical studies on the flexural behaviour of circular CFST members, the combination of multi-axial stress states, namely the concrete confinement effects and the steel bi-axial stress state, was found to benefit the bending capacity of CFST columns. Therefore, the bending capacity of the CFST columns will be underestimated by the Distributed Plasticity (DP) model when uni-axial material properties are used. To account for the material strength variation under multi-axial stress states and to improve the accuracy of the DP model, material strength correction methods were proposed based on the analytical results given by the 3D Finite Element (FE) model. A comprehensive parametric study, which involved a wide range of diameter-thickness ratios, material properties and axial load levels, had been carried out to study the influence of CFST cross-section properties on the multi-axial stress states. Based on the parametric study, the confined concrete strength under bending was found to be independent on the constant axial load level. On the contrary, the steel bi-axial stress was proved to be sensitive to the constant axial load level. A section factor $\xi$ was used to represent both the geometry and material properties of a CFST cross-section. With statistic operations, $\xi$-dependent correction equations were derived to account for the material strength variation under multi-axial stress states. The feasibility of the proposed material strength correction equations was verified to be good for both the design and modelling of circular CFST members.
	\end{abstract}

	\section{Introduction}
	The behaviour of concrete-filled steel tubes (CFSTs) is governed by the contribution of the steel tube and the concrete core. As concluded from previous experimental studies (Silva et al. 2016 \cite{Silva2016} and Silva et al. 2017 \cite{RN140}), The concrete core constrains the steel tube to its interior, thus delaying the occurrence of the local buckling to higher levels of deformation and the ductility of the steel tube is improved. Besides its excellent ductility, the composition of concrete and steel will also cause the two components under multi-axial stress state and thus, lead to the material properties variance. The steel tube provides the confined stress to the concrete core and make it under tri-axial stress state, which is well known as concrete confinement effect. The strength and ductility of the concrete core are enhanced by the confinement effect, therefore, the capacity of CFST member under either compression or bending is benefited. In recent years, there were many researchers had investigated the concrete confinement effect of CFST members. Based on the test results, Sakino et al. 2004 \cite{RN41}, Fujimoto et al. 2004 \cite{RN15} and Inai et al. 2004 \cite{RN30} had tested circular and rectangular CFST columns under axial compressive loading, eccentric compressive loading and bending, respectively. All the researchers figured out that, the circular CFST members had a more significant concrete confinement effect than the rectangular columns. Hu et al. 2003 \cite{RN1} and Hu et al. 2005 \cite{RN29} had derived the equations to calculate the confined concrete strength under either compression or bending. The concrete confinement effect of CFST member was found to be sensitive to its diameter-thickness ratio. Regarding Eurocode 4 \cite{RN64}, the concrete confinement effect of circular CFST members is considered in both compressive and bending cases.
	\par
	Due to the expansion of the concrete core, the steel tube of CFST will be under bi-axial stress state when the member is loaded. The bi-axial stress state will have an influence on the steel yield stress. According to the research works of Fujimoto et al. 2004 \cite{RN15} and Inai et al. 2004 \cite{RN30}, the bi-axial stress state would increase the steel yield stress when the CFST member is under tension while descrese the steel yield stress when the member is compressed. Regarding the Eurocode 4 \cite{RN64} design provision, it considers a reduction of the steel yield stress when calculating the circular CFST column's plastic resistance to compression. While for the consideration of the flexural resistance of the circular CFST member, the Eurocode 4 is very restrictive since the code does not consider any effects of steel bi-axial stress state to exist.
	\par
	The Distributed Plasticity (DP) model is a widely used modelling choice for CFST columns. As the DP model is a simplified numerical solution coupled with fibre element, it could not capture the multi-axial stress states, for both concrete core and steel tube. Therefore, the material properties of CFST should be corrected when applying the DP model. In recent studies of CFST columns, all the researchers (Han 2004 \cite{RN21}, Fujimoto et al. 2004 \cite{RN15}, Inai et al. 2004 \cite{RN30}, Liang et al. 2006 \cite{RN34}, Valipour and Foster 2010 \cite{RN50}, Tort and Hajjar 2010 \cite{RN46}, Zubydan and ElSabbagh 2011 \cite{RN60}, Lai and Varma 2016 \cite{RN32}, etc.) who worked with DP model had considered the concrete strength under confinement effect. But regarding the steel bi-axial stress, only Fujimoto et al. 2004 \cite{RN15}, Inai et al. 2004 \cite{RN30} and Lai and Varma 2016 \cite{RN32} had considered the its influence on the steel yield stress during modelling. The neglecting of steel bi-axial stress states of CFST member may miss leading the analytical results and decrease the accuracy of the DP model. 
	\par
	As one may have inferred, there is a clear gap in the consideration of multi-axial stress states, including both concrete confinement effect and steel bi-axial stress state, when model or design the CFST members under flexural loads. The research presented in this paper aims to address this limitation of both the literature and the current design methodologies. This paper starts with a brief description of the multi-axial stress states and the verification of the previous proposed material strength correction method for CFSTs. Based on the 3D Finite Element (FE) model proposed in Jiang et al. 2018 [?], New material strength correction methods, which aims to overcome the aforementioned limitations of the DP model, are developed and validated. By conducting a parametric study on the flexural behaviour of circular CFST members, closed-form expressions which focus on the calculation of the confined concrete stress and the steel bi-axial steel bi-axial yield stress are derived and their feasibility is verified against experimental data.
	
	\section{Multi-axial Stress States of Circular CFST Members}
	\subsection{Multi-axial Stress States}
	When the CFST members are under loading, the composite effects are introduced due to the interaction between concrete core and steel tube. Thus, both the concrete and steel are under the multi-axial stress states. The confined stress provided by the steel tube makes the concrete core under tri-axial stress state, and thus, caused the concrete confinement effect, which can improve its the strength and ductility. Thus, the concrete confinement effect will benefit the flexural capacity of CFST members. There were many researchers studied the concrete confinement effect for CFST members in recent years (Han et al. 2001 \cite{RN23}, Hu et al. 2003 \cite{RN1}, Hu et al. 2005 \cite{RN29}, Tort and Hajjar 2010 \cite{RN46} and Lai and Varma 2016 \cite{RN32}) .
	\par
	Not only the concrete core, but also the steel tube is under multi-axial stress state. The concrete core expands under axial load and the hoop stress (or circumferential stress) is generated around the steel tube. As the thickness of the steel tube is relatively small compared to the cross-section diameter, the steel tube could be considered as a shell component. Therefore, the material is under bi-axial stress state (Kwan et al. 2016 \cite{RN31}). Figure \ref{fig:1} shows the stress state of the steel tube of circular CFST member. Since the CFST member is under bending, part of the steel tube is under compression and part of it is under tension and the influences of bi-axial stress state on the two parts are different. As can be seen in Figure \ref{fig:1}, for the compression area, the steel is combined with the compressive stress on vertical direction ($\sigma_2$) and tensile stress on the horizontal direction ($\sigma_1$). While for the tension area, the steel is under tensile stress on both vertical and horizontal directions.
	\par
	\begin{figure}[h]
		\centering
		\includegraphics{Figures/Figure_1.pdf}
		\caption{Steel bi-axial stress state under bending force}
		\label{fig:1}
	\end{figure}
	\par
	To consider the steel yield stress variance under bi-axial stress state, the yield surface is introduced (Figure \ref{fig:2}). The positive/negative values on the two axes stand for the tensile/compressive stress, respectively. The hoop stress ($\sigma_h$) is plotted on the \textit{x} axis and the corresponding steel yield stresses under compression ($f_{yc}$) and tension ($f_{yt}$) are annotated on the yield surface. It could be found that, the steel compressive yield stress ($f_{yc}$) is smaller than the uni-axial yield stress ($f_y$), which will have negative influence on the member flexural capacity. On the contrary, the steel tensile yield stress ($f_{yt}$) is larger than the uni-axial yield stress ($f_y$), which will benefit the member strength under bending. For circular CFST member under centrally compression, the whole steel section is under compression. With the influence of bi-axial stress state, the steel yield stress reduces from $f_y$ to $f_{yc}$. For this reason, there is a reduction factor introduced in Eurocode 4 \cite{RN64} to reduce the steel yield stress when considering the plastic resistance of circular CFST to compression. Regarding the flexural capacity of CFST members, the influence of the steel bi-axial stress state on it depends on the relative areas between the compressive/tensile regions as well as the values of $f_{yt}$ and $f_{yc}$.
	\par
	\begin{figure}[h]
		\centering
		\includegraphics{Figures/Figure_2.pdf}
		\caption{Steel yield surface}
		\label{fig:2}
	\end{figure}
	\par
	The previous research (Jiang et al. 2018 [?]) had carried out a numerical study on the influence of the multi-axial stress states on the bending capacity of circular CFST columns. In the research work, a comprehensive 3D Finite Element (FE) model, which can capture the multi-axial stress states, and a Distributed Plasticity (DP) model, which cannot consider the multi-axial stress states influence, were developed and validated by experimental results from Silva et al. 2016 \cite{Silva2016}. The 3D FE model was proved to be accurate on the flexural behaviour simulation of the circular CFST member, as the solid and shell elements used by the model can consider the concrete and steel strength variation under multi-axial stress states. But regarding the DP model, its predicted capacity was found to be always smaller than the test results. Considering that the effects caused by multi-axial stress states could not be simulated by the fibre element, the researchers concluded that the combination of multi-axial stress states could benefit the bending behaviour of circular CFST member under bending combined with low/moderate constant axial load.
	\par
	To further verify the influence of the steel bi-axial stress state on material strength, the stress versus strain curves of the steel elements located on the same height level are extracted and plotted in Figure \ref{fig:3}. The steel uni-axial stress/strain curve is also plotted in the same figure as the reference curve. The tensile strain/stress are defined as positive values. It could be seen that, the steel compressive yield stress values are less than the uni-axial yield stress and the tensile yield strain is larger than the uni-axial yield stress, which enhances the conclusion obtained from the yield surface.
	\par
	\begin{figure}[h]
		\centering
		\includegraphics{Figures/Figure_4.pdf}
		\caption{The vertical stress/strain curves output from the 3D FE model}
		\label{fig:3}
	\end{figure}
	\par
	\subsection{Previous Studies on Multi-axial Stress State Influence}
	In the past, there were many researchers had carried out the studies on the prediction of confined concrete strength for CFST members (Han et al. 2001 \cite{RN23}, Hu et al. 2003 \cite{RN1}, Hu et al. 2005 \cite{RN29} and Tort and Hajjar 2010 \cite{RN46} e.t.c). But in these researches, the influence of steel bi-axial stress state was ignored. Therefore, these derived confined concrete strength equations may result in the inaccuracy of the numerical prediction. Lai and Varma 2016 \cite{RN32} had performed a parametric study on CFST members under compressive loading with the consideration of multi-axial stress states. The equation to calculate the confined concrete strength had been derived. Moreover, the steel compressive yield strength in the research was reduced to $0.9f_y$. But regarding the steel tensile yield stress, there was no correction. Thus, the proposed material strength correction equations of Lai and Varma 2016 \cite{RN32} were suitable for CFST members subjected to axial compression but not sufficient for members under bending.
	\par
	Inai et al. 2004 \cite{RN30} and Fujimoto et al. 2004 \cite{RN15} had used the DP model to simulate the bending behaviour of CFST members. To consider the influences of multi-axial stress states, they combined the previous researches on the confined concrete strength and the steel bi-axial yield stress (Kato and Nishiyama 1980 \cite{RN139}, Sakino 1994 \cite{RN138}, Nakahara et al. 1999 \cite{RN137} and Sakino et al. 1998 \cite{RN135}) and summarized the material strength correction equations for both concrete and steel. For circular CFST members, the confined concrete strength is given in Equation (\ref{eq:1}).
	\par
	\begin{equation}
	f_{cc} = f_c + 4.1 \frac{0.38tf_y}{D-2t}
	\label{eq:1}
	\end{equation}
	\par
	Regarding the steel bi-axial stress state, a constant hoop stress, which equals to $0.19f_y$, was assumed and the steel yield stresses under compression and tension were adopted as $0.91f_y$ and $1.08f_y$, respectively. The proposed DP models from Jiang et al. 2018[?] were re-analysed with the material strengths corrected by these equations to verify their accuracy. Table 1 shows the confined concrete strength (fcc) and the corrected steel yield stresses (fyc and fyt).
	
	
	
	
	
	\bibliography{../../BibTex/Mine,../../BibTex/CFST,../../BibTex/Eurocode,../../BibTex/Hollow_Sections,../../BibTex/Concrete}
	
	\bibliographystyle{unsrt}
	
\end{document}