\documentclass[12pt, A4]{article}
\usepackage{geometry}
\geometry{margin=1in}

\title{Influence of Multi-Axial Stress States on the Flexural Behaviour of Circular Concrete-Filled Steel Tubes}
\author{Yadong Jiang \and António Silva \and Bledar Kalemia \and José Miguel Castro \and Ricardo Monteiro}


\date{}

\begin{document}
	\maketitle
	
	\begin{abstract}
		The paper proposes the methodologies and the equations to consider the influence of multi-axial stress states on the material strength of circular Concrete-filled Steel Tubular (CFST) columns. From the previous experimental and numerical studies on the flexural behaviour of circular CFST members, the combination of multi-axial stress states, namely the concrete confinement effects and the steel bi-axial stress state, was found to benefit the bending capacity of CFST columns. Therefore, the bending capacity of the CFST columns will be underestimated by the Distributed Plasticity (DP) model when uni-axial material properties are used. To account for the material strength variation under multi-axial stress states and to improve the accuracy of the DP model, material strength correction methods were proposed based on the analytical results given by the 3D Finite Element (FE) model. A comprehensive parametric study, which involved a wide range of diameter-thickness ratios, material properties and axial load levels, had been carried out to study the influence of CFST cross-section properties on the multi-axial stress states. Based on the parametric study, the confined concrete strength under bending was found to be independent on the constant axial load level. On the contrary, the steel bi-axial stress was proved to be sensitive to the constant axial load level. A section factor $\xi$ was used to represent both the geometry and material properties of a CFST cross-section. With statistic operations, $\xi$-dependent correction equations were derived to account for the material strength variation under multi-axial stress states. The feasibility of the proposed material strength correction equations was verified to be good for both the design and modelling of circular CFST members.
	\end{abstract}

	\section{Introduction}
	The behaviour of concrete-filled steel tubes (CFSTs) is governed by the contribution of the steel tube and the concrete core. As concluded from previous experimental studies (Silva et al. 2016 \cite{Silva2016} and Silva et al. 2017 \cite{RN140}), The concrete core constrains the steel tube to its interior, thus delaying the occurrence of the local buckling to higher levels of deformation and the ductility of the steel tube is improved. Besides its excellent ductility, the composition of concrete and steel will also cause the two components under multi-axial stress state and thus, lead to the material properties variance. The steel tube provides the confined stress to the concrete core and make it under tri-axial stress state, which is well known as concrete confinement effect. The strength and ductility of the concrete core are enhanced by the confinement effect, therefore, the capacity of CFST member under either compression or bending is benefited. In recent years, there were many researchers had investigated the concrete confinement effect of CFST members. Based on the test results, Sakino et al. 2004 \cite{RN41}, Fujimoto et al. 2004 \cite{RN15} and Inai et al. 2004 \cite{RN30} had tested circular and rectangular CFST columns under axial compressive loading, eccentric compressive loading and bending, respectively. All the researchers figured out that, the circular CFST members had a more significant concrete confinement effect than the rectangular columns. Hu et al. 2003 \cite{RN1} and Hu et al. 2005 \cite{RN29} had derived the equations to calculate the confined concrete strength under either compression or bending. The concrete confinement effect of CFST member was found to be sensitive to its diameter-thickness ratio. Regarding Eurocode 4 \cite{RN64}, the concrete confinement effect of circular CFST members is considered in both compressive and bending cases.
	\par
	Due to the expansion of the concrete core, the steel tube of CFST will be under bi-axial stress state when the member is loaded. The bi-axial stress state will have an influence on the steel yield stress. According to the research works of Fujimoto et al. 2004 \cite{RN15} and Inai et al. 2004 \cite{RN30}, the bi-axial stress state was found to have a positive effect on the steel tensile yield stress and a negative influence on the steel  compressive yield stress. Regarding the Eurocode 4 \cite{RN64} design provision, it considers a reduction of the steel yield stress when calculating the circular CFST column's plastic resistance to compression. While for the consideration of the flexural resistance of the circular CFST member, the Eurocode 4 is very restrictive since the codes does not consider any effect of steel bo-axial stress state to exist.
	\par
	Besides the 3D FE model, the Distributed Plasticity (DP) model was also a widely used modelling choice for CFST columns (Han 2004 [13], Tort and Hajjar 2010 [14], Zubydan and ElSabbagh 2011 [15], etc.). As the DP model is a simplified numerical solution coupled with fibre element, it could not capture the multi-axial stress states, for both concrete core and steel tube. Therefore, the material properties of CFST should be corrected when applying the DP model. All the researchers (Han 2004 [13], Fujimoto et al. 2004 \cite{RN15}, Inai et al. 2004 \cite{RN30}, Valipour and Foster 2010 [16], Liang et al. 2006 [17], Tort and Hajjar 2010 [14], Zubydan and ElSabbagh 2011 [15], Lai and Varma 2016 \cite{RN32}, etc.) who worked with DP model had considered the concrete strength under confinement effect. But regarding the steel bi-axial stress, only Fujimoto et al. 2004 \cite{RN15}, Inai et al. 2004 \cite{RN30} and Lai and Varma 2016 \cite{RN32} had considered the its influence on the steel yield stress during modelling. The neglecting of steel bi-axial stress states of CFST member may miss leading the analytical results and decrease the accuracy of the DP model.
	\par
	As one may have inferred, there is a clear gap in the consideration of multi-axial stress states, including both concrete confinement effect and steel bi-axial stress state, on CFST members under flexural loads. The research presented in this paper aims to address this limitation of both the literature and the current design methodologies. This paper starts with a brief description of the multi-axial stress states and the verification of previous proposed methods on material strength correction for CFSTs. New strength correction methods based on the 3D FE modelling approach, which aims to overcome the limitations of the DP model, was developed and validated. By conducting a parametric study on the flexural behaviour of CFST members, closed-form expressions which focus on the calculation the concrete confinement stress and effective steel strength were derived and their feasibility was verified against experimental data. 
	
	\bibliography{../../BibTex/Mine,../../BibTex/CFST,../../BibTex/Eurocode}
	
	\bibliographystyle{unsrt}
	
\end{document}