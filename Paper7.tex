\documentclass[12pt, A4]{article}
\usepackage{geometry}
\geometry{margin=1in}

\title{Influence of Multi-Axial Stress States on the Flexural Behaviour of Circular Concrete-Filled Steel Tubes}
\author{Yadong Jiang \and António Silva \and Bledar Kalemia \and José Miguel Castro \and Ricardo Monteiro}


\date{}

\begin{document}
	\maketitle
	
	\begin{abstract}
		The paper proposes the methodologies and the equations to consider the influence of multi-axial stress states on material strength of circular Concrete-filled Steel Tubular (CFST) columns. From the previous experimental and numerical studies on the flexural behaviour of circular CFST members, the combination of multi-axial stress states, namely the concrete confinement effects and the steel bi-axial stress states, was found to benefit the bending capacity of CFST columns. Therefore, the bending capacity of the CFST columns will be underestimated by the Distributed Plasticity (DP) model when uni-axial material properties are used. To account for the material strength variation under multi-axial stress states and to improve the accuracy of the DP model, material strength correction methods were proposed based on the analytical results given by the 3D Finite Element (FE) model. A comprehensive parametric study, which involved a wide range of diameter-thickness ratios, material properties and axial load levels, had been carried out to study the influence of CFST cross-section properties on the multi-axial stress states. Based on the parametric study, the confined concrete strength under bending was found to be independent on the constant axial load level. On the contrary, the steel bi-axial stress was proved to be sensitive to the constant axial load level. A section factor $\xi$ was used to represent both the geometry and material properties of a CFST cross-section. With statistic operations, $\xi$-dependent correction equations were derived to account for the material strength variation under multi-axial stress states. The feasibility of the proposed material strength correction equations was verified to be good for both the design and modelling of circular CFST members.
	\end{abstract}

	\section{Introduction}
	The behaviour of concrete-filled steel tubes (CFSTs) is governed by the contribution of the steel tube and the concrete core. As concluded from previous experimental studies (\cite{})The concrete core constrains the steel tube to its interior, thus delaying the occurrence of the local buckling to higher levels of deformation and the ductility of the steel tube is improved. Besides its excellent ductility, the composition of concrete and steel will also cause the two components under multi-axial stress state and thus, lead to the material properties change. The steel tube provides the confined stress to the concrete core and make it under tri-axial stress state, which is well known as concrete confinement effect. The strength and ductility of the concrete core are enhanced by the confinement effect, therefore, the capacity of CFST member under either compression or bending is benefited. In recent years, there were many researchers had investigated the concrete confinement effect of CFST members. Based on the test results, Sakino et al. 2004 [2], Fujimoto et al. 2004 [3] and Inai et al. 2004 [4] had tested circular and rectangular CFST columns under axial compressive loading, eccentric compressive loading and bending, respectively. All the researchers figured out that, the circular CFST members had a more significant concrete confinement effect than the rectangular columns. Hu et al. 2003 [5] and Hu et al. 2005 [6] had derived the equations to calculate the confined concrete strength under either compression or bending. The concrete confinement effect of CFST member was found to be sensitive to its diameter-thickness ratio.
	\par	
	Due to the expansion of concrete core, the steel tube of CFST will also be under bi-axial stress state when loaded. The multi-axial stress state will change the steel yield stress under compression and tension. In the research works of Fujimoto et al. 2004 [3], Inai et al. 2004 [4] and Lai and Varma 2016 [7], a reduction of steel yield stress had been made when the CFST was under compression. While for the tensile yield stress of steel, Fujimoto et al. 2004 [3] and Inai et al. 2004 [4]considered that the bi-axial stress state would have a positive influence on the tensile yield stress of the steel, and thus, an amplification on its value was made.
	To accurately predict the behaviour of CFST member, the multi-axial stress states should be considered during numerical modelling. The 3D Finite Element (FE) model was a numerical model commonly used by many researchers (Hu et al. 2003 [5], Ellobody et al. 2006 [8], Duarte et al. 2016 [9], Lai and Varma 2016 [7] etc.). Since the 3D FE model consists with 3D/2D elements, it could take the material strength variation under complex axial stress states into consideration. Therefore, the model was proved to be accurate in simulating the response of CFST members under different loading types, namely under axial loading (Hu et al. 2003 [5], Ellobody et al. 2006 [8], Duarte et al. 2016 [9], etc.), under monotonic bending (Hu et al. 2005 [6], Wang et al. 2014 [10], etc.) and under cyclic bending (Goto et al. 2010 [11] and Imani et al. 2015 [12], etc.). 
	
	Besides the 3D FE model, the Distributed Plasticity (DP) model was also a widely used modelling choice for CFST columns (Han 2004 [13], Tort and Hajjar 2010 [14], Zubydan and ElSabbagh 2011 [15], etc.). As the DP model is a simplified numerical solution coupled with fibre element, it could not capture the multi-axial stress states, for both concrete core and steel tube. Therefore, the material properties of CFST should be corrected when applying the DP model. Almost all the researchers (Han 2004 [13], Fujimoto et al. 2004 [3], Inai et al. 2004 [4], Valipour and Foster 2010 [16], Liang et al. 2006 [17], Tort and Hajjar 2010 [14], Zubydan and ElSabbagh 2011 [15], Lai and Varma 2016 [7], etc.) who worked with DP model had considered the concrete strength under confinement effect. But regarding the steel bi-axial stress, only Fujimoto et al. 2004 [3], Inai et al. 2004 [4] and Lai and Varma 2016 [7] had considered the its influence on the steel yield stress during modelling. The neglecting of steel bi-axial stress states of CFST member may miss leading the analytical results and decrease the accuracy of the DP model.
	Regarding the Eurocode 4 [18] (EC4) design provision, the confined concrete strength and the corrected steel yield stress are considered. However, the code clearly targets this consideration for member under compression, whilst being very restrictive for bending combined with compression. When it comes to fully flexural loads, the codes does not consider any effect of multi-axial stress states to exist.
	As one may have inferred, there is a clear gap in the consideration of multi-axial stress states, including both concrete confinement effect and steel bi-axial stress state, on CFST members under flexural loads. The research presented in this paper aims to address this limitation of both the literature and the current design methodologies. This paper starts with a brief description of the multi-axial stress states and the verification of previous proposed methods on material strength correction for CFSTs. New strength correction methods based on the 3D FE modelling approach, which aims to overcome the limitations of the DP model, was developed and validated. By conducting a parametric study on the flexural behaviour of CFST members, closed-form expressions which focus on the calculation the concrete confinement stress and effective steel strength were derived and their feasibility was verified against experimental data. 
	
	
	
\end{document}