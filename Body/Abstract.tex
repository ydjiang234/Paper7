\abstract
The behaviour of concrete-filled steel tubes (CFSTs) is governed by the contribution of the steel tube and the concrete core. Their interaction will lead to the development of multi-axial stress states, namely the concrete confinement effect and the steel bi-axial stress state. From the previous experimental and numerical studies on the flexural behaviour of circular CFST members, the combination of multi-axial stress states was found to benefit the bending capacity of CFST columns. Therefore, the bending capacity of the CFST columns will be underestimated when the Distributed Plasticity (DP) model with uni-axial material properties are used. To account for the material strength under multi-axial stress state and improve the accuracy of DP model, strength correction methods were proposed based on the analytical results of 3D Finite Element (FE) model. Based on the material strength correction methods, a comprehensive parametric study, which involved a large range of CFST cross-section Diameter-thickness ratios, material properties and axial load levels, had been carried out. A section factor $\xi$ was used to represent both the geometry and material properties of CFST cross-section. According to the parametric study, the confined concrete strength under bending was found to be independent on the constant axial load level. On the contrary, the steel bi-axial stress was proved to be sensitive on the constant axial load level. With statistic operations, $\xi$-dependent correction equations were derived to account for the material strength variation under multi-axial stress states. The feasibility of the material strength correction equations was verified to be good for both the design and the modelling of circular CFST members.
