Concrete-filled steel tube (CFST) members have been increasingly used in practical design throughout the globe, as it is recognized that such combination of concrete and steel is generally more efficient and economical seismic-resistant solution, in comparison with, for example, reinforced concrete members. The concrete core constrains the steel tube to its interior, thus delaying the occurrence of the local buckling to higher levels of deformation, while the steel tube may allow for the development of confinement effects on the concrete core, thus benefiting both the capacity and ductility of CFSTs.
Regarding the modelling of CFSTs, two numerical modelling approaches have seen widespread use in the scientific community, namely distributed plasticity modelling and 3D finite element modelling. The former is usually combined with fibre elements at multiple sections of the member, (e.g. Aval et al. 2002 [1], Varma et al. 2002 [2], Inai et al. 2004 [3], Han 2004 [4], Liang et al. 2006 [5], Chung 2010 [6], Zubydan and ElSabbagh 2011 [7]), and is commonly validated against experimental data. In order to account for the interaction between the concrete core and the steel tube, advanced fibre element types were derived by different authors, namely modified fibre elements (Valipour and Foster 2010 [8]) and mixed fibre elements (Hajjar et al. 1998 [9], Tort and Hajjar 2009 [10], Tort and Hajjar 2010 [11], Denavit and Hajjar 2011 [12]). However, local buckling effects are not usually captured, and, in some cases, may be critical. As for 3D finite element modelling, it poses the advantage of accounting directly for the interaction between the components of the CFST members, whilst also capturing the development of local buckling effects. However, it is a computationally-heavier approach. Notwithstanding, it has been used to represent the behaviour of CFST members under compression (e.g. Hu et al. 2003 [13], Ellobody et al. 2006 [14], Tao et al. 2013 [15]), bending combined with constant axial load (e.g. Han et al. 2008 [16], Goto et al. 2011 [17], Wang et al. 2014 [18]) and torsion (Han et al. 2007 [19]).
As previously stated, the effect of confinement is one of the key benefits associated with CFST members. On this topic, Hu et al. 2003 [13] carried-out a comprehensive research study on the confinement effect for both circular and rectangular CFST members under compression, deriving closed-form expressions to calculate the confined concrete strength based on the member geometry and material strengths. The paper clearly indicated that these effects are much more substantial for circular members than for rectangular members. The European design code (EN 2004 [20]) follows a similar approach, allowing for the consideration of the confinement effect for design purposes on circular members only. However, the code clearly targets this consideration for members under compression, whilst being very restrictive for bending combined with compression. When it comes to fully flexural loads, the code does not consider any effect of confinement to exist.
As one may have inferred, there is a clear gap in the consideration of multi-axial stress states on CFST members under flexural loads. The research presented in this paper aims to address this limitation of both the literature and the current design methodologies. This paper starts with a brief description of an experimental campaign of CFST columns under monotonic lateral loading combined or not with axial loads. The influence of multi-axial stress states is discussed by comparing the test results with the expected member capacities according to Eurocode 4. In order to have a deep understanding of the multi-axial stress states, two numerical models, namely a 3D finite element model in Abaqus 2014 [21]and a distributed plasticity model in OpenSees (Mazzoni et al. 2006 [22]), were developed and calibrated against experimental data. A strength correction method based on the 3D finite element modelling approach, which aims to overcome the limitations of the distributed plasticity model, was developed and validated. By conducting a parametric study on the flexural behaviour of CFST members, closed-form expressions which focus on the calculation the concrete confinement stress and effective steel strength were derived and their feasibility was verified against experimental data. 
