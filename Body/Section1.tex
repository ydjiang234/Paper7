Concrete-filled steel tube (CFST) members have been increasingly used in practical design throughout the globe, as it is recognized that such combination of concrete and steel is generally more efficient and economical seismic-resistant solution, in comparison with, for example, reinforced concrete members (Morino et al. 2001). The concrete core constrains the steel tube to its interior, thus delaying the occurrence of the local buckling to higher levels of deformation and the ductility of the steel tube is improved. Besides the ductility, the composition of concrete and steel will also cause both of the two components under multi-axial stress state and thus, leads to the material properties change. Regarding the concrete core, the steel tube provides the confined stress to it and make it under tri-axial stress state, which is well known as concrete confinement effect. The strength and ductility of the concrete core are enhanced by the confinement effect, which will benefits the capacity of CFST member under either compression or bending. In recent years, there were many researchers had investigated the concrete confinement effect of CFST members. Based on the test results, Sakino et al. 2004 pointed out that, there was confinement effect existing in CFST members under compressive loading, and the circular CFST members should be a more significant concrete confinement effect than the rectangular columns. Similarly, Fujimoto et al. 2004 and Inai et al. 2004 had performed the experimental tests on the behaviour of CFST members under eccentric compressive loading and bending, respectively. And the researchers found that the concrete confinement effect could be neglected for rectangular CFST columns.

The steel tube of CFST member will also be under multi-axial stress states under loading, which is caused by the expansion of concrete core. The multi-axial stress states will make the steel have different yield stress under compression and tension. In the research works of Fujimoto et al. 2004, Inai et al. 2004 and Varmar and Lai 2016, a reduction of steel yield stress had been made when consider the capacity of CFST under compression. While for the tensile yield stress of steel, Fujimoto et al. 2004 and Inai et al. 2004 considered that the multi-axial stress states would have a positive influence on the tensile yield stress of the steel, and thus, an amplification on it was made.
To accurately predict the behaviour of CFST member, the multi-axial stress states should be considered during the modelling. The 3D Finite Element (FE) model was a commonly used numerical model by many researchers (Hu et al. 2003, Ellobody et al. 2006, Wang et al. 2014, Varma and Lai 2016 etc.). Since the 3D FE model used 3D/2D elements, it could consider the material strength change under complex axial stress states. Therefore, the model was proved to be accurate in simulating the response of CFST members under different loading types, namely under axial load (Hu et al. 2003, Ellobody et al. 2005 and Duarte et al. 2016), under monotonic bending (Hu et al. 2005,nad Wang et al. 2014) and under cyclic bending (Goto et al. 2010and Imano et al. 2015). 

Besides the 3D FE model, the Distributed Plasticity (DP) model was also a common choice of modelling CFST columns (Aval et al. 2002, Han et al. 2004, Chung 2010, Zubydan and ElSabbagh 2011 etc.). As the DP model is a simplified numerical solution with fibre element represented as the CFST cross-section, it could not capture the multi-axial stress states, namely the concrete confinement effect and the steel bi-axial stress states. Therefore, the material properties should be corrected when applicating the DP model. Nearly all the researchers (Han et al. 2004, Valipour and Foster 2010, Zubydan and ElSabbagh 2011, Tort and Hajjar 2010, Liang et al. 2006) had considered the concrete confinement effects during the modelling, but regarding the steel bi-axial stress.The neglecting of steel bi-axial stress states of CFST member during the behaviour study of CFST member may miss leading the analytical results and decrease the accuracy of the DP model. On the contray, Fujimoto et al. 2004, Inai et al. 2004 and Varma and Lai 2016 had considered both concrete confinement effect and the steel bi-axial stress states, the feasibility of these methods are detailed in Section 2.

Regarding the Eurocode 4 design provision, the multi-axial stress states of circular CFST member was considered during the calculation of its plastic resistance capacity under axial compression. The confined concrete strength and the corrected steel yield stress are considered in the code. However, the code clearly targets this consideration for member under compression, whilst being very restrictive for bending combined with compression. When it comes to fully flexural loads, the codes does not consider any effect of confinement to exist.   
As one may have inferred, there is a clear gap in the consideration of multi-axial stress states, including both concrete confinement effect and steel bi-axial stress state, on CFST members under flexural loads. The research presented in this paper aims to address this limitation of both the literature and the current design methodologies. This paper starts with a brief description of the multi-axial stress states and the verification of previous proposed material strength correction methods. A strength correction method based on the 3D FE modelling approach, which aims to overcome the limitations of the DP model, was developed and validated. By conducting a parametric study on the flexural behaviour of CFST members, closed-form expressions which focus on the calculation the concrete confinement stress and effective steel strength were derived and their feasibility was verified against experimental data. 
